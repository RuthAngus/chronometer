\documentclass[useAMS, usenatbib, preprint, 12pt]{aastex}
\usepackage{cite, natbib}
\usepackage{float}
\usepackage{epsfig}
\usepackage{cases}
\usepackage[section]{placeins}
\usepackage{graphicx, subfigure}
\usepackage{color}
\usepackage{bm}
\usepackage{enumerate}
\usepackage{pdflscape}
\usepackage{eqnarray}

\newcommand{\racomment}[1]{{\color{red}#1}}

% Affiliations
\newcommand{\columbia}{1}
\newcommand{\simons}{2}
\newcommand{\nyu}{3}
\newcommand{\cca}{4}
\newcommand{\mpia}{5}

% Missions
\newcommand{\Kepler}{{\it Kepler}}
\newcommand{\kepler}{\Kepler}
\newcommand{\kic}{\KIC}
\newcommand{\KIC}{\KIC}
\newcommand{\corot}{{\it CoRoT}}
\newcommand{\Ktwo}{{\it K2}}
\newcommand{\ktwo}{\Ktwo}
\newcommand{\TESS}{{\it TESS}}
\newcommand{\tess}{{\it TESS}}
\newcommand{\LSST}{{\it LSST}}
\newcommand{\lsst}{{\it LSST}}
\newcommand{\Wfirst}{{\it Wfirst}}
\newcommand{\wfirst}{{\it Wfirst}}
\newcommand{\SDSS}{{\it SDSS}}
\newcommand{\sdss}{{\it SDSS}}
\newcommand{\PLATO}{{\it PLATO}}
\newcommand{\plato}{{\it PLATO}}
\newcommand{\TGAS}{{\it TGAS}}
\newcommand{\tgas}{{\it TGAS}}
\newcommand{\APOGEE}{{\it APOGEE}}
\newcommand{\apogee}{{\it APOGEE}}
\newcommand{\gaia}{{\it Gaia}}
\newcommand{\Gaia}{{\it Gaia}}
\newcommand{\rave}{{\it RAVE}}
\newcommand{\Rave}{{\it RAVE}}

% Stellar parameters
\newcommand{\Teff}{$T_{\mathrm{eff}}$}
\newcommand{\teff}{$T_{\mathrm{eff}}$}
\newcommand{\FeH}{[Fe/H]}
\newcommand{\feh}{[Fe/H]}
\newcommand{\Av}{$A_v$}
\newcommand{\av}{$A_v$}
\newcommand{\logg}{log \emph{g}}
\newcommand{\dnu}{$\Delta \nu$}
\newcommand{\numax}{$\nu_{\mathrm{max}}$}

% Common expressions
\newcommand{\ie}{{\it i.e.}}
\newcommand{\eg}{{\it e.g.}}
\newcommand{\etal}{{\it et al.}}

% Paper-specific
\newcommand{\nkictgas}{2000}

\begin{document}

\title{A coherent and comprehensive dating system for stars in the Milky Way}

\author{%
    Ruth Angus\altaffilmark{\columbia,}\altaffilmark{\simons},
   David Hogg\altaffilmark{\nyu,}\altaffilmark{\cca,}\altaffilmark{\mpia},
    \etal
   % Melissa Ness\altaffilmark{\mpia},
   % David Kipping\altaffilmark{\columbia}, \etal
}

\altaffiltext{\columbia}{Department of Astronomy, Columbia
University, NY, NY}
\altaffiltext{\simons}{Simons Fellow, RuthAngus@gmail.com}
\altaffiltext{\nyu}{Center for Cosmology and Particle Physics, New York
University, NY, NY}
\altaffiltext{\mpia}{Max Planck Institute of Astronomy, Heidelberg, Germany}
\altaffiltext{\cca}{Center for Computational Astrophysics, Flatiron Institute,
NY, NY}

\begin{abstract}
% Aims
Stellar age underpins a wide range of astronomical applications but
remains one of the most challenging stellar properties to infer.
% Method
Competing dating methods with different observational inputs often produce
    wildly inconsistent age estimates.
In this work we attempt to combine multiple methods into one coherent model.
For the first time, we combine apparent magnitudes and colour information,
    typically used to infer an age using stellar evolutionary tracks, with
    other age proxies: rotation period and vertical action.
We identify a sample of stars with rotation periods, colours, spectroscopic
    parameters and kinematics to calibrate this dating method.
% Results
\end{abstract}


\section{Introduction}
\label{sec:intro}

The processes behind the formation of the Milky Way and the objects within it
are some of the most elusive and complicated topics in astronomy today.
These topics are connected by a common theme: stellar ages.
Stellar ages provide the key to understanding the evolution of all
astrophysical objects.
For main sequence (MS) stars however, age is a difficult property to infer.
This is predominantly because hydrogen burning stars do not change appreciably
during their time spent on the MS: a star like the Sun will grow in
luminosity by around a factor of two before becoming a red giant.
Luminosity is therefore not a very sensitive probe of age.
As explained further later in this paper: the Sun's rotation period will vary
by almost an order of magnitude over its MS lifetime, thus rotation
period is a much more sensitive age proxy than luminosity or temperature.

In addition to the difficulties imposed by the slow timescale for variability
withing MS stars, different dating methods often produce inconsistent
predictions for the age of a star.
For example, an asteroseismic age will not necessarily agree with a isochronal
or rotational age.
This is in part because the underlying processes generating the evolution of
the observable properties are different and in part because our understanding
of the underlying physics is flawed or incomplete.
The various available dating methods can be categorised by the underlying
physical process they trace.
For example, evolutionary models track the radial extent of the
hydrogen-burning core and age-rotation relations model the evolving state of
the internal magnetic dynamo.
In addition, dating methods can be classified by their level of empiricism,
\ie\ the number of free parameters that need to tuned when fitting the models
to the data.
The physics behind the evolving luminosity and effective temperature of a star
as a result of core hydrogen burning is, for example, very well understood and
does not need calibrating; physics determine these models.
On the other hand, magnetic activity evolution is poorly understood and must
be calibrated using available data.
In table \ref{tab:dating_methods} we provide an overview of various dating
methods, the main observables associated with them, the underlying physics
driving the changing observables, the types of star the method applies to and
the empirical or physical nature of the model.

\begin{landscape}
\begin{table}
\begin{center}
\caption{A list of all dating methods with the types of stars they apply to
    and their effective or physical natures.}
\begin{tabular}{lcccc}
    Method & Observable & Underlying cause & Applicable to... & Effective or
    Physical model? \\
    \hline
    Rotation & $P_{\mathrm{rot}}$ & Magnetic activity & $<$ 4 Gyr, MS &
    Effective \\
    Activity & Radio, H$\alpha$ \& X-ray & Magnetic activity & $<$
    5 Gyr, low mass & Effective \\
    Time variability & Time variability & ? & ? & ? \\
    \hline
    Isochrones & $T_{\mathrm{eff}}$, $M_v$ & Core/shell fusion &
    Subgiants/giants & Physical (model dependent) \\
    Spectroscopic twins & Spectra & ? & ? & ? \\
    \hline
    Asteroseismology & Frequencies & Core/shell fusion & Old, low
    mass, Giants & Physical (model dependent) \\
    \hline
    White dwarf cooling & $M_v$ & Cooling & WDs (\& companions) &
    Physical (1 free parameter) \\
    \hline
    $J_z$, $J_{\phi}$, $J_r$ & Position \& velocity & Dynamic heating & Disk
    stars & Physical (1 free parameter) \\
    Galactocentric position & Position & Formation history &
    All stars & Effective \\
    Coeval pairs & Position \& velocity & Disruption & $<$ 1 Gyr, disk &
    0 free parameters \\
    Open Clusters & Disruption & Positions \& velocities & $<$ 2 Gyr &
    0 free parameters \\
    Exoplanet dynamics & Stability \& eccentricity & Interactions & $<$ 500
    Myrs & 0 free parameters \\
    \hline
    C/N ratio & C/N ratio & Convective processes & Red giants, A \& F &
    Effective \\
    Chemical abundances & Abundances & ISM enrichment & $>$ 8 Gyr?
    & Effective \\
    Lithium abundance & L abundance & $L$ depletion & $<$ 500
    Myrs? & Effective \\
    $[Y/Mg]$ & $[Y/Mg]$ & Enrichment & ? & Effective \\
    \hline
    Nuclear isotopes & ? & ? & ? & ? \\
    Unusual HRD objects & $T_{\mathrm{eff}}$, $M_v$ & Mergers, ? & Special
    cases & ? \\
    Universe age & All of the above & Finite age of the Universe &
    All stars & 0 free parameters \\
\end{tabular}
\end{center}
\end{table}
\label{tab:dating_methods}
\end{landscape}

In this paper we focus on four of these dating methods: rotation,
asteroseismology, stellar evolution models and galactic kinematics.
A brief overview of each of these methods is provided in the following
sections.

\subsection{Rotation-Dating}
\label{sec:rotation}

Main sequence (MS) stars comprise the majority of our galaxy but their ages
are notoriously difficult to measure.
Their positions on the HR diagram don't change significantly during their
    hydrogen burning lifetimes, a fact that is convenient for life on Earth
    but inconvenient for galactic archaeologists.
Now, due to the abundance of rotation periods for MS stars provided by Kepler
    and to-be provided by TESS, LSST and Wfirst, rotation-dating is the most
    readily available, precise method for inferring stellar ages.
Rotation-dating works well for young stars but a question mark still hangs
    over its accuracy for stars older than the Sun.
Recent results show that old \kepler\ asteroseismic stars rotate more rapidly
    than expected given their age \citep[\eg][]{Angus2015, Vansaders2016,
    Metcalfe2016}.
This has been attributed to an evolving magnetic dynamo: as stars reach a
    critical Rossby number (the ratio of rotation period to the convective
    overturn timescale), their magnetic field `switches off' and stars
    maintain a consistent rotation period after that time.
Whilst this physical explanation produces a model that fits the data, it
    is driven by observations, not theory, and other explanations could
    provide an answer.
The data sets typically used to test the age-rotation relations are highly
    heterogeneous and each set has its own detection and selection biases.
For example, asteroseismology favours quiet stars whereas rotation periods are
    easiest to measure for active stars.

% A history of rotation-dating.
The phenomenon of magnetic braking in MS stars was first observed almost fifty
years ago by \citet{Skumanich1972} who observed that the rotation periods of
the Sun and young cluster stars seemed to decay with the square-root of time.
Later, a mass-dependence was added to the relation between age and rotation
period --- less massive stars lose angular momentum faster than more massive
ones.
\citet{Kawaler1988} derived a formalism for this angular momentum loss and his
relation depended on the mass loss rate, the ....
More recently, \citet{Barnes2003} demonstrated that a simple relation could be
used to describe `gyrochronology', the method of rotation-dating, and further
works \citep[\eg][]{Barnes2007, Mamajek2008, Barnes2010, Meibom2011},
continue to demonstrate that the relation between rotation period and age
holds true while theorists \citep[\eg][]{Matt2012, Epstein2014} modify and
extend the efforts to produce physical models of this phenomenon.

Not only do a number of dating methods exists, several different models are
often available for the same dating method.
In the case of rotation-dating...

\racomment{What happens when you can't measure a rotation period?}
\racomment{What about the difference in physics that you are investigating?}

\subsection{Asteroseismology}

\subsection{Stellar Evolution models}
The ongoing physical processes in the core of a star is reflected externally
by an increase in luminosity and temperature.
As hydrogen is converted to helium via nuclear fusion in the core, the mass
fraction, etc, etc.
Leading to etc, etc.


\subsection{Galactic Dynamics}

This paper is laid out as follows.
% In \textsection \ref{sec:dating_methods} we provide an overview of each dating
% method used in this paper and demonstrate their inconsistency in age
% predictions.
In \textsection \ref{sec:data} we describe the data used to calibrate our new
model, which is described in \textsection \ref{sec:method}.
In \textsection \ref{sec:results} we present a new catalogue of ages.

\section{Data}
\label{sec:data}
We assembled a sample of stars that have at least two of the four dating
methods available.
Our main asteroseismic sample targets comprise the $~$500 short cadence
\kepler\ prime targets.
These stars all have light curves and therefore rotation periods (or potential
rotation periods).
The majority are also in Gaia DR1, therefore have parallaxes, positions and
2-dimensional velocities, which can provide constraints on their vertical
actions.
A subset of these stars have spectroscopic information and therefore make
excellent isochronal targets.
Our main rotation sample is the \nkictgas\ stars in both the \kepler\ input
catalogue (KIC) and \TGAS.
These stars have potential rotation periods, parallaxes, positions and
2-dimensional velocities.
Some of these stars, particularly those with confirmed planets, also have
spectroscopic information and could provide isochronal ages.
Our main isochronal data set is the intersection of the \apogee\ and \rave\
catalogues with \tgas.
Relatively precise isochronal ages can be inferred from the spectroscopic
stellar properties of these stars, the majority which have ages from {\it the
cannon} \citep{Ness2015, Casey2016}.
They also have \gaia\ parallaxes, positions and proper motions as well as
radial velocities, providing vertical actions calculated from the full
3-dimensional phase-space.

\begin{landscape}
\begin{table}
\begin{center}
\caption{A summary of the data sets used in this project}
\begin{tabular}{lccc}
    Sample & Catalogue intersection &  Methods available \\
    \hline
    Asteroseismic & \citet{Chaplin2014} \& TGAS & Asteroseismology, rotation,
    isochrones, dynamics \\
    Rotational & KIC \& TGAS & Rotation, isochrones, dynamics \\
    Isochronal & RAVE, APOGEE \& TGAS & Isochrones, dynamics \\
\end{tabular}
\end{center}
\end{table}
\label{tab:data}
\end{landscape}

\section{Method}
\label{sec:method}

We combine the following dating methods: gyrochronology, isochrone fitting,
asteroseismology and galactic kinematics into one model.
In classical isochrone fitting, one might calculate the probability of age
($A$), mass ($M$), \feh, distance ($D$) and extinction (\av), given a set of
apparent magnitudes in different filters, \eg\ $M_V$, $M_J$ and $M_K$
\begin{equation}
    p(A, M, [Fe/H], D, A_v | M_V, M_J, M_K) \propto p(M_v, M_J, M_K |
    A, M, [Fe/H], D, A_v)p(A, M, [Fe/H], D, A_v).
\end{equation}
Note that there are no free parameters in the model here --- the model is
entirely governed by physics and does not need to be calibrated further.
Similarly, in classical gyrochronology, one might calculate the probability of
age given rotation period and $B-V$ colour,
\begin{equation}
    p(A | P_{rot}, B-V) \propto p(P_{rot}, B-V | A)p(A)
\end{equation}
In a gyrochronology model, because our understanding of the physics is
incomplete, there are some free parameters that must be calibrated.
In the \citet{Barnes2003} gyrochronology parameterisation, these are $a$, $b$
and $n$, $A = P^{1/n} - (B-V-0.4)^{b/n}$.
During a calibration exercise, the observables, $A$, $P_{rot}$ and $B-V$ will
be known and the parameters, $a$, $b$ and $n$ will be inferred,
\begin{equation}
    p(a, b, n | A, P_{rot}, B-V) \propto p(A, P_{rot}, B-V | a,
    b, n) p(a, b, n).
\end{equation}
The vertical action-age relation also has free parameters that must be
calibrated using observations, $A = \alpha (\sigma_{Jz}^2)^\beta$.
These free parameters, $\alpha$ and $\beta$ can, again be calibrated via
\begin{equation}
    p(\alpha, \beta | \{A\}, \{J_z\}) \propto p(\{A\}, \{J_z\} | \alpha,
    \beta) p(\alpha, \beta).
\end{equation}

The unique aspect of our analysis is to combine inference over ages with
calibration of the free parameters in the gyrochronology and vertical
action-age relations:
\begin{eqnarray}
    &&p(\{A\}, \{M\}, \{[Fe/H]\}, \{D\}, \{A_v\}, a, b, n, \alpha, \beta|
    \{M_V\}, \{M_J\}, \{M_K\}, \{P_{rot}\}, \{B-V\}, \{J_z\}) \\
    \propto &&p(\{M_v\}, \{M_J\}, \{M_K\}, \{P_{rot}\},
    \{B-V\}, \{J_z\} | \{A\}, \{M\}, \{[Fe/H]\}, \{D\}, \{A_v\}, a, b, n,
    \alpha, \beta) \\
    &&p(A, M, [Fe/H], D, A_v, a, b, n, \alpha, \beta).
\end{eqnarray}
We can use conditional independence to break this down further into the
following:
\begin{eqnarray}
    &&p(\{A\}, \{M\}, \{[Fe/H]\}, \{D\}, \{A_v\}, a, b, n, \alpha, \beta|
    M_V, M_J, M_K,
    \{P_{rot}\}, \{B-V\}, \{J_z\}) \\
    \propto &&p(\{M_v\}, \{M_J\}, \{M_K\} | \{A\}, \{M\}, \{[Fe/H]\}, \{D\},
    \{A_v\})
    p(A)p(M)p([Fe/H])p(D)p(A_v) \\
    &&p(\{P_{rot}\}, \{B-V\}, a, b, n | A)p(a)p(b)p(n) \\
    &&p(\{J_z\}, \alpha, \beta | \{A\}) p(\alpha)p(\beta).
\end{eqnarray}
In other words
\begin{eqnarray}
    &&p(\mathrm{Ages~\&~other~parameters}|\mathrm{observables}) \\
    \propto &&\mathcal{L}_{\mathrm{isochronal~age}} \times
    \mathcal{L}_{\mathrm{gyro~age}} \times
    \mathcal{L}_{\mathrm{kinematic~age}} \\
    && \times p(\mathrm{Ages~\&~other~parameters})
\end{eqnarray}.


As an example, take the Sun: a star with temperature $5777K$, $ \log(g) =
4.44$ $[Fe/H] = 0.0$ and $P_{rot} = 26$ days.

In order to establish the level of inconsistency between dating methods, we
first compare predictions for each of the above data sets.
This is purely an illustrative test as these dating methods are not always
independent, for example many rotation-dating models are calibrated using
asteroseismic age predictions.
For the asteroseismic sample we find that

\section{Results and Discussion}
\label{sec:results}

\section{Conclusions}

% acknowledgements
This research was funded by the Simons Foundation.
Some of the data presented in this paper were obtained from the Mikulski
Archive for Space Telescopes (MAST).
STScI is operated by the Association of Universities for Research in
Astronomy, Inc., under NASA contract NAS5-26555.
Support for MAST for non-HST data is provided by the NASA Office of Space
Science via grant NNX09AF08G and by other grants and contracts.
This paper includes data collected by the Kepler mission. Funding for the
Kepler mission is provided by the NASA Science Mission directorate.
\racomment{Add Gaia sprint, TGAS \& Apogee acknowledgements.}

\bibliographystyle{plainnat}
\bibliography{chronometer}

\end{document}
