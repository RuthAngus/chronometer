\documentclass[useAMS, usenatbib, preprint, 12pt]{aastex}
\usepackage{cite, natbib}
\usepackage{float}
\usepackage{epsfig}
\usepackage{cases}
\usepackage[section]{placeins}
\usepackage{graphicx, subfigure}
\usepackage{color}
\usepackage{bm}
\usepackage{enumerate}
\usepackage{pdflscape}

\newcommand{\columbia}{1}
\newcommand{\simons}{2}
\newcommand{\nyu}{3}
\newcommand{\cca}{4}
\newcommand{\mpia}{5}
\newcommand{\Kepler}{{\it Kepler}}
\newcommand{\kepler}{\Kepler}
\newcommand{\corot}{{\it CoRoT}}
\newcommand{\Ktwo}{{\it K2}}
\newcommand{\ktwo}{\Ktwo}
\newcommand{\TESS}{{\it TESS}}
\newcommand{\LSST}{{\it LSST}}
\newcommand{\Wfirst}{{\it Wfirst}}
\newcommand{\SDSS}{{\it SDSS}}
\newcommand{\PLATO}{{\it PLATO}}
\newcommand{\Teff}{$T_{\mathrm{eff}}$}
\newcommand{\teff}{$T_{\mathrm{eff}}$}
\newcommand{\FeH}{[Fe/H]}
\newcommand{\feh}{[Fe/H]}
\newcommand{\ie}{{\it i.e.}}
\newcommand{\eg}{{\it e.g.}}
\newcommand{\etal}{{\it et al.}}
\newcommand{\logg}{log \emph{g}}
\newcommand{\dnu}{$\Delta \nu$}
\newcommand{\numax}{$\nu_{\mathrm{max}}$}

\begin{document}

\title{A coherent and comprehensive dating system for stars in the Milky Way}

\author{%
    Ruth Angus\altaffilmark{\columbia,}\altaffilmark{\simons},
   David Hogg\altaffilmark{\nyu,}\altaffilmark{\mpia,}\altaffilmark{\cca},
    \etal
   % Melissa Ness\altaffilmark{\mpia},
   % David Kipping\altaffilmark{\columbia}, \etal
}

\altaffiltext{\columbia}{Department of Astronomy, Columbia
University, NY, NY}
\altaffiltext{\simons}{Simons Fellow, RuthAngus@gmail.com}
\altaffiltext{\nyu}{Center for Cosmology and Particle Physics, New York
University, NY, NY}
\altaffiltext{\mpia}{Max Planck Institute of Astronomy, Heidelberg, Germany}
\altaffiltext{\cca}{Center for Computational Astrophysics, Flatiron Institute,
NY, NY}

\begin{abstract}
Words go here.
\end{abstract}

\section{Introduction}
\label{sec:intro}

The processes behind the formation of the Milky Way and the objects within it
are some of the most elusive and complicated topics in astronomy today,
connected by a common theme: stellar ages.
Ages provide the key to understanding the evolution of all astrophysical
objects, but age, unfortunately, one of the most difficult stellar properties
to measure.
Different dating methods often produce inconsistent predictions for the age of
the same star.
This is in part because the underlying processes generating the evolution of
the observable properties are different and in part because our understanding
of the underlying physics is flawed or incomplete.
In table \ref{tab:dating_methods} we present a list of dating methods, the
main observables associated with that method, the underlying physics driving
the changing observables, the types of star the method applies to and the
empirical or physical nature of the model.

\begin{landscape}
\begin{table}
\begin{center}
\caption{A list of all dating methods with the types of stars they apply to
    and their effective or physical natures.}
\begin{tabular}{lcccc}
    Method & Observable & Underlying cause & Applicable to... & Effective or
    Physical model? \\
    \hline
    Rotation & $P_{\mathrm{rot}}$ & Magnetic activity & $<$ 4 Gyr, MS &
    Effective \\
    Activity & Radio, H$\alpha$ \& X-ray & Magnetic activity & $<$
    5 Gyr, low mass & Effective \\
    Time variability & Time variability & ? & ? & ? \\
    Isochrones & $T_{\mathrm{eff}}$, $M_v$ & Core/shell fusion &
    Subgiants/giants & Physical (model dependent) \\
    Asteroseismology & Frequencies & Core/shell fusion & Old, low
    mass, Giants & Physical (model dependent) \\
    White dwarf cooling & Cooling & $M_v$ & WDs (\& companions) &
    Physical (1 free parameter) \\
    $J_z$, $J_{\phi}$, $J_r$ & Position \& velocity & Dynamic heating & Disk
    stars & Physical (1 free parameter) \\
    Galactocentric position & Position & Formation history &
    All stars & Effective \\
    Coeval pairs & Position \& velocity & Disruption & $<$ 1 Gyr, disk &
    0 free parameters \\
    Open Clusters & Disruption & Positions \& velocities & $<$ 2 Gyr &
    0 free parameters \\
    Exoplanet dynamics & Stability \& eccentricity & Interactions & $<$ 500
    Myrs & 0 free parameters \\
    Universe age & All of the above & Finite age of the Universe &
    All stars & 0 free parameters \\
    C/N ratio & C/N ratio & Convective processes & Red giants, A \& F &
    Effective \\
    Chemical abundances & Abundances & ISM enrichment & $>$ 8 Gyr?
    & Effective \\
    Lithium abundance & L abundance & $L$ depletion & $<$ 500
    Myrs? & Effective \\
    Nuclear isotopes & ? & ? & ? & ? \\
    Unusual HRD objects & $T_{\mathrm{eff}}$, $M_v$ & Mergers, ? &
    Special cases & ? \\
\end{tabular}
\end{center}
\end{table}
\label{tab:dating_methods}
\end{landscape}

\subsection{Rotation-Dating}
\label{sec:rotation}

Main sequence (MS) stars comprise the majority of our galaxy but their ages
are notoriously difficult to measure.
Their positions on the HR diagram don't change significantly during their
    hydrogen burning lifetimes, a fact that is convenient for life on Earth
    but inconvenient for galactic archaeologists.
Now, due to the abundance of rotation periods for MS stars provided by Kepler
    and to-be provided by TESS, LSST and Wfirst, rotation-dating is the most
    readily available, precise method for inferring stellar ages.
Rotation-dating works well for young stars but a question mark still hangs
    over its accuracy for stars older than the Sun.
Recent results show that old \kepler\ asteroseismic stars rotate more rapidly
    than expected given their age \citep[\eg][]{Angus2015, Vansaders2016,
    Metcalfe2016}.
This has been attributed to an evolving magnetic dynamo: as stars reach an
    critical Rossby number (the ratio of rotation period to the convective
    overturn timescale), their magnetic field `switches off' and stars
    maintain a consistent rotation period after that time.
Whilst this physical explanation produces a model that fits the data, it
    is driven by observations, not theory, and other explanations could
    provide an answer.
The data sets typically used to test the age-rotation relations are highly
    heterogeneous and each set has its own detection and selection biases.
For example, asteroseismology favours quiet stars whereas rotation periods are
    easiest to measure for active stars.

% A history of rotation-dating.
The phenomenon of magnetic braking in MS stars was first observed almost fifty
years ago by \citet{Skumanich1972} who observed that the rotation periods of
the Sun and young cluster stars seemed to decay with the square-root of time.
Later, a mass-dependence was added to the relation between age and rotation
period --- less massive stars lose angular momentum faster than more massive
ones.
\citet{Kawaler1988} derived a formalism for this angular momentum loss and his
relation depended on the mass loss rate, the ....
More recently, \citet{Barnes2003} demonstrated that a simple relation could be
used to describe `gyrochronology', the method of rotation-dating, and further
works \citep[\eg][]{Barnes2007, Mamajek2008, Barnes2010, Meibom2011},
continue to demonstrate that the relation between rotation period and age
holds true while theorists \citep[\eg][]{Matt2012, Epstein2014} modify and
extend the efforts to produce physical models of this phenomenon.

\section{Method}

\section{Results and Discussion}

\section{Conclusions}

% acknowledgements
This research was funded by the Simons Foundation.
Some of the data presented in this paper were obtained from the Mikulski
Archive for Space Telescopes (MAST).
STScI is operated by the Association of Universities for Research in
Astronomy, Inc., under NASA contract NAS5-26555.
Support for MAST for non-HST data is provided by the NASA Office of Space
Science via grant NNX09AF08G and by other grants and contracts.
This paper includes data collected by the Kepler mission. Funding for the
Kepler mission is provided by the NASA Science Mission directorate.

\bibliographystyle{plainnat}
\bibliography{chronometer}

\end{document}
